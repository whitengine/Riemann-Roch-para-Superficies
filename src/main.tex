%\documentclass[english,10pt,draft]{amsart}
%\documentclass[english,10pt]{amsart}

\documentclass[spanish,12pt]{amsart}
\usepackage{fouriernc}%la fuente
%\usepackage[english]{babel}
\usepackage[spanish]{babel}

\relpenalty=9999
\binoppenalty=9999
%\mathsurround=1pt %espacio antes y después de una fórmula en el texto


%\usepackage[matrix,arrow]{xy}
%\xyoption{all}

\usepackage{amscd,amssymb,amsfonts,amsmath}
\usepackage{tikz-cd}

\usepackage{subfiles} %esto es para modularizar el overleaf
%para usar este paquete solamente hay que usar el comando
%\subfile{}

\graphicspath{{./}} %esto es para que encuentre las figuras hechas con pdf_tex en inkscape
\usepackage{subfig}
\usepackage{caption} %es para captionoffigure
\usepackage{xparse}
\usepackage{xstring}

\usepackage{bm}
\usepackage{graphicx}
%%\usepackage{graphics}
\usepackage{epsfig}
\usepackage{mathrsfs}
\usepackage{xcolor}
\DeclareMathAlphabet{\mathscrbf}{OMS}{mdugm}{b}{n}

\usepackage{epigraph}

\usepackage{braket} %para definir \set , \Set y que los conjuntos se vean mas lindos

\usepackage[shortlabels]{enumitem}

\definecolor{violet}{rgb}{0.0,0.2,0.7}
\definecolor{rouge2}{rgb}{0.8,0.0,0.2}
\usepackage{hyperref}
\hypersetup{
    %bookmarks=true,         % show bookmarks bar?
    unicode=false,          % non-Latin characters in Acrobat’s bookmarks
    pdftoolbar=true,        % show Acrobat’s toolbar?
    pdfmenubar=true,        % show Acrobat’s menu?
    pdffitwindow=false,     % window fit to page when opened
    pdfstartview={FitH},    % fits the width of the page to the window
    pdftitle={},    % title
    pdfauthor={},     % author
    colorlinks=true,       % false: boxed links; true: colored links
   linkcolor=violet,          % color of internal links
    citecolor=rouge2,        % color of links to bibliography
    filecolor=black,      % color of file links
    urlcolor=cyan}           % color of external links


%\renewcommand{\thepart}{\Roman{part}}
\setcounter{tocdepth}{1} %tableofcontents

\unitlength=1cm

\usepackage[text={6.7in,9.2in},centering]{geometry}

\makeatletter
\renewcommand\subsection{\@startsection{subsection}{2}%
  \z@{.5\linespacing\@plus.7\linespacing}{-.5em}%
  %{\normalfont\scshape}}
  %{\normalfont\itshape}}
  {\normalfont\sffamily}}
\makeatother

%\renewcommand{\baselinestretch}{1.1}
%\renewcommand{\baselinestretch}{1.2}


\newcommand{\cqfd}{\hfill $\square$}
\newcommand{\ie}{\textit{i.e. }}

\newcommand{\Hilb}{\textup{Hilb}}
\newcommand{\Pic}[1]{\textup{Pic}{ \: (#1)}}
\newcommand{\Alb}{\textup{Alb}}
\newcommand{\Spec}{\textup{Spec}}
\newcommand{\Proj}{\textup{Proj}}
\newcommand{\Supp}{\textup{Supp}}
\newcommand{\Exc}{\textup{Exc}}
\newcommand{\Rat}{\textup{RatCurves}}
\newcommand{\RatCurves}{\textup{RatCurves}}
\newcommand{\RC}{\textup{RatCurves}^n}
\newcommand{\Univ}{\textup{Univ}}
\newcommand{\Chow}{\textup{Chow}}
\newcommand{\Sing}{\textup{Sing}}
\newcommand{\CS}{\textup{CS}}
\newcommand{\codim}{\textup{codim}}
\newcommand{\Res}{\textup{Res}}
\newcommand{\modulo}{\textup{mod}}




%\let \cedilla =\c
%\renewcommand{\c}[0]{{\mathbb C}}
%\newcommand{\p}[0]{{\mathbb P}}


\newcommand{\Nef}{\textup{Nef}}
%\renewcommand{\Big}{\textup{Big}}
\renewcommand{\div}{\textup{div}}
\newcommand{\Div}{\textup{Div}}
\newcommand{\Amp}{\textup{Amp}}
\newcommand{\Pef}{\textup{Pef}}
\newcommand{\Eff}{\textup{Eff}}
\newcommand{\NS}{\textup{N}^1}
\newcommand{\N}{\textup{N}}
\newcommand{\Z}{\textup{Z}}
\renewcommand{\H}{\textup{H}}
\newcommand{\h}{\textup{h}}
\newcommand{\discrep}{\textup{discrep}}

\newcommand{\NE}{\overline{\textup{NE}}}
\newcommand{\Hol}{\textup{Hol}}

\newcommand{\into}{\hookrightarrow}
\newcommand{\map}{\dashrightarrow}
\newcommand{\lra}{\longrightarrow}

\renewcommand{\le}{\leqslant}
\renewcommand{\ge}{\geqslant}
\newcommand{\Diff}{\textup{Diff}}
\newcommand{\D}{\Delta}
\renewcommand{\d}{\delta}
\newcommand{\G}{\Gamma}
\newcommand{\Fix}{\textup{Fix}}
\newcommand{\Bs}{\textup{Bs}}
\newcommand{\mult}{\textup{mult}}
\newcommand{\Mob}{\textup{Mob}}
\newcommand{\B}{\textup{B}}

\newcommand{\bA}{\textbf{A}}
\newcommand{\bB}{\textup{\textbf{A}}}
\newcommand{\bC}{\textup{\textbf{C}}}
\newcommand{\bD}{\textbf{D}}
\newcommand{\bE}{\textbf{E}}
\newcommand{\bF}{\textbf{F}}
\newcommand{\bG}{\textbf{G}}
\newcommand{\bH}{\textbf{H}}
\newcommand{\bI}{\textbf{I}}
\newcommand{\bJ}{\textbf{J}}
\newcommand{\bK}{\textbf{K}}
\newcommand{\bL}{\textbf{L}}
\newcommand{\bM}{\textbf{M}}
\newcommand{\bN}{\textbf{N}}
\newcommand{\bO}{\textbf{O}}

%\newcommand{\bP}{\mathbb{P}}
\newcommand{\bQ}{\mathbb{Q}}
%\newcommand{\bZ}{\mathbb{Z}}

\newcommand{\bR}{\textup{\textbf{R}}}
\newcommand{\bS}{\textbf{S}}
\newcommand{\bT}{\textbf{T}}
\newcommand{\bU}{\textbf{U}}
\newcommand{\bV}{\textbf{V}}
\newcommand{\bW}{\textbf{W}}
\newcommand{\bX}{\textbf{X}}
\newcommand{\bY}{\textbf{Y}}
\newcommand{\bZ}{\textbf{Z}}
\renewcommand{\bB}{\textbf{B}}
\newcommand{\bP}{\textbf{P}}


\newcommand{\cA}{\mathcal{A}}
\newcommand{\cB}{\mathcal{B}}
\newcommand{\cC}{\mathcal{C}}
\newcommand{\cD}{\mathcal{D}}
\newcommand{\cE}{\mathcal{E}}
\newcommand{\cF}{\mathcal{F}}
\newcommand{\cG}{\mathcal{G}}
\newcommand{\cH}{\mathcal{H}}
\newcommand{\cI}{\mathcal{I}}
\newcommand{\cJ}{\mathcal{J}}
\newcommand{\cK}{\mathcal{K}}
\newcommand{\cL}{\mathcal{L}}
\newcommand{\cM}{\mathcal{M}}
\newcommand{\cN}{\mathcal{N}}
\newcommand{\cO}{\mathcal{O}}
\newcommand{\cP}{\mathcal{P}}
\newcommand{\cQ}{\mathcal{Q}}
\newcommand{\cR}{\mathcal{R}}
\newcommand{\cS}{\mathcal{S}}
\newcommand{\cT}{\mathcal{T}}
\newcommand{\cU}{\mathcal{U}}
\newcommand{\cV}{\mathcal{V}}
\newcommand{\cW}{\mathcal{W}}
\newcommand{\cX}{\mathcal{X}}
\newcommand{\cY}{\mathcal{Y}}
\newcommand{\cZ}{\mathcal{Z}}

\newcommand{\frA}{\mathfrak{A}}
\newcommand{\frB}{\mathfrak{B}}
\newcommand{\frC}{\mathfrak{C}}
\newcommand{\frD}{\mathfrak{D}}
\newcommand{\frE}{\mathfrak{E}}
\newcommand{\frF}{\mathfrak{F}}
\newcommand{\frG}{\mathfrak{G}}
\newcommand{\frH}{\mathfrak{H}}
\newcommand{\frI}{\mathfrak{I}}
\newcommand{\frJ}{\mathfrak{J}}
\newcommand{\frK}{\mathfrak{K}}
\newcommand{\frL}{\mathfrak{L}}
\newcommand{\frM}{\mathfrak{M}}
\newcommand{\frN}{\mathfrak{N}}
\newcommand{\frO}{\mathfrak{O}}
\newcommand{\frP}{\mathfrak{P}}
\newcommand{\frQ}{\mathfrak{Q}}
\newcommand{\frR}{\mathfrak{R}}
\newcommand{\frS}{\mathfrak{S}}
\newcommand{\frT}{\mathfrak{T}}
\newcommand{\frU}{\mathfrak{U}}
\newcommand{\frV}{\mathfrak{V}}
\newcommand{\frW}{\mathfrak{W}}
\newcommand{\frX}{\mathfrak{X}}
\newcommand{\frY}{\mathfrak{Y}}

%\renewcommand{\frm}{\mathfrak{m}} % accolades dans xymatrix

\newcommand{\frm}{\mathfrak{m}}
\newcommand{\frf}{\mathfrak{f}}
\newcommand{\frg}{\mathfrak{g}}
\newcommand{\frt}{\mathfrak{t}}


\newcommand{\sA}{\mathscr{A}}
\newcommand{\sB}{\mathscr{B}}
\newcommand{\sC}{\mathscr{C}}
\newcommand{\sD}{\mathscr{D}}
\newcommand{\sE}{\mathscr{E}}
\newcommand{\sF}{\mathscr{F}}
\newcommand{\sG}{\mathscr{G}}
\newcommand{\sH}{\mathscr{H}}
\newcommand{\sI}{\mathscr{I}}
\newcommand{\sJ}{\mathscr{J}}
\newcommand{\sK}{\mathscr{K}}
\newcommand{\sL}{\mathscr{L}}
\newcommand{\sM}{\mathscr{M}}
\newcommand{\sN}{\mathscr{N}}
\newcommand{\sO}{\mathscr{O}}
\newcommand{\sP}{\mathscr{P}}
\newcommand{\sQ}{\mathscr{Q}}
\newcommand{\sR}{\mathscr{R}}
\newcommand{\sS}{\mathscr{S}}
\newcommand{\sT}{\mathscr{T}}
\newcommand{\sU}{\mathscr{U}}
\newcommand{\sV}{\mathscr{V}}
\newcommand{\sW}{\mathscr{W}}
\newcommand{\sX}{\mathscr{X}}
\newcommand{\sY}{\mathscr{Y}}
\newcommand{\sZ}{\mathscr{Z}}

\newcommand{\sbfE}{\mathscrbf{E}}
\newcommand{\sbfG}{\mathscrbf{G}}
\newcommand{\sbfH}{\mathscrbf{H}}
\newcommand{\sbfL}{\mathscrbf{L}}
\newcommand{\sbfN}{\mathscrbf{N}}

\newcommand{\art}{\textup{Art}}
\newcommand{\ens}{\textup{Ens}}
\newcommand{\sch}{\textup{Sch}}
\newcommand{\Der}{\textup{Der}}



\newtheorem{theorem}{Teorema}[section]
\newtheorem*{theorema}{Teorema A}
\newtheorem*{theoremb}{Teorema B}
\newtheorem*{theoremc}{Teorema C}
\newtheorem*{theoremd}{Teorema D}
\newtheorem*{theoreme}{Teorema E}
\newtheorem{maintheorem}[theorem]{Teorema Principal}
\newtheorem{question}[theorem]{Pregunta}
\newtheorem{lemma}[theorem]{Lema}
\newtheorem{corollary}[theorem]{Corolario}
\newtheorem{corollaries}[theorem]{Corolarios}
\newtheorem{proposition}[theorem]{Proposición}
\newtheorem{criteria}[theorem]{Criterio}
\newtheorem{conjecture}[theorem]{Conjectura}
\newtheorem{principle}[theorem]{Principio}

\newtheorem*{theorem*}{Teorema}
\theoremstyle{definition}
\newtheorem{definition}[theorem]{Definición}
\newtheorem{condition}[theorem]{Condición}
\newtheorem{say}[theorem]{}
\newtheorem{hint}[theorem]{Hint}
\newtheorem{trick}[theorem]{Truco}
\newtheorem{exercise}[theorem]{Ejercicio}
\newtheorem{problem}[theorem]{Problema}
\newtheorem{construction}[theorem]{Construcción}
\newtheorem{algorithm}[theorem]{Algoritmo}
\newtheorem{obs}[theorem]{Observación}
\newtheorem{observations}[theorem]{Observaciones}
%\renewcommand{\theremark}{}
\newtheorem{note}[theorem]{Nota}            %\renewcommand{\thenote}{}
\newtheorem{summary}[theorem]{Resumen}         %\renewcommand{\thesumm}{}
\newtheorem{acknowledgement}{Agradecimientos}
\newtheorem{notation}[theorem]{Notación}
\newtheorem{atention}[theorem]{Atención}
\newtheorem{definition-theorem}[theorem]{Definición-Teorema}
\newtheorem{definition-lemma}[theorem]{Definición-Lema}
\newtheorem{convention}[theorem]{Convención}
\newtheorem{application}[theorem]{Aplicación}



\theoremstyle{remark}
\newtheorem*{claim}{Claim}
%\newtheorem*{rem}{Remark}
%\newtheorem*{rems}{Remarks}
\newtheorem{case}{Caso}
\newtheorem{subcase}{Subcaso}
\newtheorem{step}{Paso}
\newtheorem{approach}{Enfoque}
%\newtheorem{principle}{Principle}
\newtheorem{fact}[theorem]{Hecho}
\newtheorem{subsay}{}
\newtheorem*{notation-and-definition}{Notación y definición}
\newtheorem{assumption}[theorem]{Hipótesis}
\newtheorem{remark}[theorem]{Observación}
\newtheorem{remarks}[theorem]{Observaciones}
\newtheorem{example}[theorem]{Ejemplo}


\numberwithin{equation}{section}

\def\factor#1.#2.{\left. \raise 2pt\hbox{$#1$} \right/\hskip -2pt\raise -2pt\hbox{$#2$}}
































%%%%%%%%%%%%%%%%%%%%%%%%%%%%%%%%%%%%%%%%%%%%%%%%%%%%%%%%%%%%%%%%%%%%%%%


%%%%%%%%%%%%%Teoría de Grupos%%%%%%%%%%%%

%Grupo simétrico de n elementos
\newcommand{\SymGrp}[1]{\mathbb{S}_{#1}}
%Grupo alternado de n elementos
\newcommand{\AltGrp}[1]{\mathbb{A}_{#1}}

%Orden de un elemento $a \in G$ de un grupo
\newcommand{\ord}[1]{\operatorname{ord} (#1)}







%%%%%%%%%%%%%%%Polinomios%%%%%%%%%%%%%%%%%%%%


%grado de una extensión algebraica
\newcommand{\degExt}[2]{[#1:#2]}
\newcommand{\degSep}[2]{[#1 : #2]_s}
\newcommand{\degInsep}[2]{[#1:#2]_i}


%espacio afin A^n
\newcommand{\afine}[1]{\mathbb{A}^{#1}}
%espacio proyectivo P^n
\newcommand{\projective}[1]{\mathbb{P}^{#1}}







%%%%%%%%%%%%%%%%%%%%%%%%%%%%%%%%%%%


%grupos de matrices
%SL
\newcommand{\SL}[2]{\operatorname{SL}_{#1} ( #2)}
%GL
\newcommand{\GL}[2]{\operatorname{GL}_{#1} ( #2)}

%matriz identidad
\newcommand{\Id}{\operatorname{Id}}



%enteros Z
\newcommand{\integers}{\mathbb{Z}}
%racionales
\newcommand{\rationals}{\mathbb{Q}}
%naturales
\newcommand{\naturals}{\mathbb{N}}
%reales R
\newcommand{\reals}{\mathbb{R}}
%imaginarios
\newcommand{\complex}{\mathbb{C}}
%p-adicos
\newcommand{\padics}{\mathbb{Q}_p}
%enteros p-adicos
\newcommand{\padicintegers}{\mathbb{Z}_p}

%cuerpos finitos
%Fp
\newcommand{\Fp}{\mathbb{F}_p}
%Fq
\newcommand{\Fq}{\mathbb{F}_q}



%valor absoluto
\newcommand{\abs}[1]{\left \vert #1 \right \vert}
%valor absoluto con dos barras
\newcommand{\Abs}[1]{\left \vert \left \vert #1 \right \vert \right \vert}

%valuacion p-adica
\newcommand{\val}[1]{\operatorname{val} (#1)}

%Hom
\newcommand{\Hom}[3]{\operatorname{Hom}_{#1} (#2, #3)}
%Hom con caligrafia cursiva
\newcommand{\HomCalli}[3]{\operatorname{\text{\calligra{Hom}}}_{\: \: \: #1} (#2, #3)}

%imagen y núcleo
\newcommand{\Imagen}{\operatorname{Im}}
\newcommand{\Ker}{\operatorname{Ker}}

%coker
\newcommand{\Coker}{\operatorname{Coker}}

%limite inverso
\newcommand{\liminv}{\varprojlim}


%un poco de typeset para categorias
\newcommand{\catname}[1]{{\operatorfont\textbf{#1}}}




%%%%%%%%%%%%%%%%%%%%%%%%%%%%%%%%%%%%%%%%%%%%%%%%%%%%%%%%%%%%%%%%%%%%%%%%%%%%%%%%%%%%%%%%%%%%%%%%%%%%%%%%%%%%%%%%%%%%%%%%%%%

\usepackage{stackengine} %%% esto es para estaquear simbolos uno arriba de otro: crear el \isomrightarrow y el \setminus mas bonito



%flecha de isomorfismo a derecha corto \isomrightarrow
\newcommand{\isomrightarrow}{\mathrel{\stackon[1pt]{$\rightarrow$}{\resizebox{!}{3pt}{$\sim$}}}}
%flecha de isomorfismo a derecha largo \isomrightarrow
\newcommand{\isomlongrightarrow}{\mathrel{\stackon[1pt]{$\longrightarrow$}{\resizebox{!}{3pt}{$\sim$}}}}




%flecha de isomorfismo a izquierda corto \isomleftarrow
\newcommand{\isomleftarrow}{\mathrel{\stackon[1pt]{$\leftarrow$}{\resizebox{!}{3pt}{$\sim$}}}}
%flecha de isomorfismo a izquierda largo \isomleftarrow
\newcommand{\isomlongleftarrow}{\mathrel{\stackon[1pt]{$\longleftarrow$}{\resizebox{!}{3pt}{$\sim$}}}}


%setminus redefinido para que se vea mas lindo
\renewcommand{\setminus}{\mathrel{\stackon[0.5pt]{}{$\smallsetminus$}}}

%%%%%%%%%%%%%%%%%%%%%%%%%%%%%%%%%%%%%%%%%%%%%%%%%%%%%%%%%%%%%%%%%%%%%%%%%%%%%%%%%%%%%%%%%%%%%%%%%%%%%%%%%%%%%%%%%%%%%%%%%%%





%flecha de gancho hook a derecha largo \hooklongrightarrow
\newcommand{\hooklongrightarrow}{\lhook\joinrel\longrightarrow}
%flecha de gancho hook a izquierda largo \hooklongleftarrow
\newcommand{\hooklongleftarrow}{\longleftarrow\joinrel\rhook}

%flecha de dos cabezas twoheads a derecha largo \twoheadlongrightarrow
\newcommand{\twoheadlongrightarrow}{\relbar\joinrel\twoheadrightarrow}
%flecha de dos cabezas twoheads a izquierda largo \twoheadlongleftarrow
\newcommand{\twoheadlongleftarrow}{\twoheadleftarrow\joinrel\relbar}



\renewcommand{\hat}[1]{\widehat{#1}}
\renewcommand{\bar}[1]{\overline{#1}}
\renewcommand{\tilde}[1]{\widetilde{#1}}

%declaro un comando nuevo para escribir restricción de funciones
\newcommand\rest[2]{{% we make the whole thing an ordinary symbol
  \left.\kern-\nulldelimiterspace % automatically resize the bar with \right
  #1 % the function
  \vphantom{\big|} % pretend it's a little taller at normal size
  \right|_{#2} % this is the delimiter
  }}


%%%%   COMANDO ALGEBRA CONMUTATIVA   %%%%

%altura de un ideal:
\newcommand{\height}{\textsc{height}}

%Clausura topológica
\newcommand{\closure}[1]{\overline{#1}}

%longitud de un A-modulo. Notacion: \length_A M
\newcommand{\length}{\operatorname{length}}

%Anulador de un $A$-módulo.
\newcommand{\Ann}[1]{\operatorname{Ann} (#1)}

%Cuerpo de fracciones. Notacion $\FracField A$.
\newcommand{\FracField}[1]{\operatorname{Fr} (#1)}


%conjunto de Lugares de un cuerpo
\newcommand{\places}[1]{\mathcal{Pl} (#1)}

%volumen
\newcommand{\Vol}[1]{\operatorname{vol}\left ( #1 \right)}


%%%%%%%%%%%%%%%%%%%%%%%%%%%%%%%%%%%%



%%%%   COMANDO ANÁLISIS  %%%%

%definimos el diferencial d de la integral "\int f(x) \dd x"
\newcommand*\dd{\mathop{}\!\mathrm{d}}

%definimos mas diferenciales
\newcommand{\dmu}[1]{\dd \mu (#1)}
\newcommand{\dnu}[1]{\dd \nu (#1)}
\newcommand{\dtheta}[1]{\dd \theta (#1)}
\newcommand{\dxi}[1]{\dd \xi (#1)}
\newcommand{\deta}[1]{\dd \eta (#1)}






%%%%   COMANDO TEORÍA DE NÚMEROS  %%%%

%Morfismo de Frobenius
\newcommand{\Frob}{\operatorname{Frob}}

%Grupo de Galois
\newcommand{\Gal}[2]{\operatorname{Gal} ( #1 / #2 )}

%Discriminante
\newcommand{\discriminant}[1]{\mathfrak{d} (#1 )}
\newcommand{\disc}{\operatorname{d}}
\newcommand{\Disc}[3]{\operatorname{D}_{#1 / #2} (#3)}

%%%%Ideales primos%%%
%escribe una letra en notación mathfrak, para denotar a un ideal o elemento primo.

\newcommand{\primo}[1]{\mathfrak{#1}}
\newcommand{\Primo}[1]{\mathfrak{\MakeUppercase{#1}}}

%anillo de enteros O_K
\renewcommand{\O}{\mathcal{O}}
%anillo de enteros con subindice de cuerpo (input, por ejemplo $K$).
\newcommand{\integralring}[1]{O_{#1}}

%caracteristica de un cuerpo Char k
\newcommand{\Char}[1]{\operatorname{Char} #1}

%traza. Notación \trace = Tr
\newcommand{\trace}{\operatorname{Tr}}

%Traza de extensiones. Notación \Tr L K \alpha = \operatorname{Tr}_{L/K} (\alpha)
\newcommand{\Tr}[1]{\operatorname{Tr}_{L/K} (#1)} %la extension es L/K por default
\newcommand{\tr}[3]{\operatorname{Tr}_{#1/#2} (#3)}

%Norma de extensiones. Notación \Norm L K \alpha = \operatorname{N}_{L/K} (\alpha)
\newcommand{\Norm}[1]{\operatorname{N}_{L/K} (#1)}%la extension es L/K por default
\newcommand{\norm}[3]{\operatorname{N}_{#1/#2} (#3)}

%%%%%%%%%%%%%%%%%%%%%%%%%%%%%%%%%%%%

%COMANDOS GEOMETRIA ALGEBRAICA

%grupo de cohomología H^n (U,V)
\renewcommand{\H}[3]{\operatorname{H}^{#1} (#2, #3)}



\usepackage{calc}


\NewDocumentCommand{\Inkscape}{O{1} m m}{
\begin{center}
\def\svgwidth{1\textwidth}
\input{#3}
\captionof{figure}{#2}
\end{center}
}









%%%DIBUJOS DE KNOTS


\usetikzlibrary{
  babel,%CAUSA PROBLEMAS PARA COMPILAR TIKZCD QUE USA "" PARA DESIGNAR CADA TEXTO DE UNA FLECHA. SE ARREGLA USANDO \usetikzlibrary{babel} despues de \usepackage{tikz-cd}.
  %
  %el resto es para agregar el QED Symbol custom como nudos
  knots,
  hobby,
  decorations.pathreplacing,
  shapes.geometric,
  calc
}

%%DIBUJA EL five knot
\newcommand{\fiveknot}{%
\begin{tikzpicture}[transform canvas={scale=0.1}]
\begin{knot}[
  consider self intersections=true,
%  draft mode=crossings,
  flip crossing/.list={2,4},
  only when rendering/.style={
%    show curve controls
  }
]
\strand[black, line width=8pt] (2,0) .. controls +(0,1) and +(54:1.0) .. (144:2) .. controls +(54:-1.0) and +(18:-1.0) .. (-72:2) .. controls +(18:1.0) and +(162:-1.0) .. (72:2) .. controls +(162:1.0) and +(126:1.0) .. (-144:2) .. controls +(126:-1.0) and +(0,-1.0) .. (2,0);
\end{knot}
\end{tikzpicture}
}


%DIBUJA EL TREFOIL KNOT
\newcommand{\trefoilknot}{%
\begin{tikzpicture}[transform canvas={scale=0.1}, line width=10pt]
\begin{knot}[
  consider self intersections=true,
%  draft mode=crossings,
  flip crossing=2,
  only when rendering/.style={
%    show curve controls
  }
  ]
\strand[black, line width=10pt] (0,2) .. controls +(2.2,0) and +(120:-2.2) .. (210:2) .. controls +(120:2.2) and +(60:2.2) .. (-30:2) .. controls +(60:-2.2) and +(-2.2,0) .. (0,2);
\end{knot}
\end{tikzpicture}
}

\DeclareCaptionFormat{custom}
{%
    \textbf{#1#2}\textit{ #3}
}
\captionsetup{format=custom}





\begin{document}

%%%% CAMBIAR EL QUED POR UN 5-KNOT
\renewcommand{\qedsymbol}{\fiveknot}

\title{Riemann-Roch para superficies}

\author{Enzo \textsc{Giannotta}}

%\address{}

%\email{}



\begin{abstract}
Presentaremos una introducción al número de intersección de dos curvas en una superficie y probaremos el Teorema de Riemann-Roch para superficies.
\end{abstract}

\maketitle

\tableofcontents
%{\small\tableofcontents}

%\epigraph{I feel that what mathematics needs least are pundits who issue prescriptions or guidelines for presumably less enlightened mortals.}{\textit{Armand Borel}}




%\subsection*{Estructura del artículo}



%%% Opcional (generalmente se agradece a gente con la quien se discutió o preguntó)
\subsection*{Agradecimientos}

Agradezco al profesor Pedro por sugerir este tema para la ayudantía, por compartir su tiempo para conversar sobre este tema, y por hacer disponible el material bibliográfico necesario para prepararlo.


\section{Notación}

Todas las superficies, que notaremos $S,S'$, serán superficies suaves proyectivas irreducibles sobre un cuerpo algebraicamente cerrado $k$. Notaremos por $D,D'$ a dos divisores de $S$; $D\sim D'$ significa que $D$ y $D'$ son linealmente equivalentes, es decir, $D-D'$ es un divisor principal. $\O_S (D)$ denota al haz invertible correspondiente al divisor $D$, y $H^i (S, \O_S (D)) =: H^i (\O_S (D)) =: H^i (D)$ a su $i$-ésimo grupo de cohomologia respecto del haz $\O_S (D)$.

Recordemos que si $X$ es una variedad algebraica proyectiva de dimensión $n$, y $\mathcal F$ es un haz coherente en $X$ (por ejemplo, en estas notas estaremos trabajando exactamente dentro de este contexto). Definimos la \textbf{característica de Euler-Poincaré} de $\mathcal F$ como la cantidad finita (gracias a los teoremas de finitud y anulación de Grothendieck; ver \cite[\S 5]{notas_pedro}):
\[
    \chi (\mathcal F) := \chi (X, \mathcal F) := \sum_{i \geq 0} (-1)^i h^i(X, \mathcal F)  = \sum_{i =0}^n (-1)^i h^i(X, \mathcal F),
\]
donde $h^i (X, \mathcal F) := h^i (\mathcal F) := \dim_k (H^i (X, \mathcal F))$. Por ejemplo, cuando $\mathcal F = \O_X (D)$ es el haz asociado a un divisor $D$ de $X$, notaremos $h^i (X, F) =: h^i (D)$; es costumbre notar $\ell (D)$ a la dimensión del \textbf{espacio de Riemann-Roch} de $D$, es decir, del espacio $H^0 (X, \O_X (D))$. Recordar que en este contexto (ver \cite[Lema 5.2.5]{notas_pedro}) la característica de Euler-Poincaré es \textbf{aditiva}, es decir, dada una sucesión exacta de haces coherentes en $X$ (variedad proyectiva)
\[
    \begin{tikzcd}
    0 \ar{r} & \mathcal F \ar{r} & \mathcal G \ar{r} & \mathcal H \ar{r} & 0
    \end{tikzcd}
\]
entonces
\[
    \chi (\mathcal G) = \chi (\mathcal F) + \chi (\mathcal H).
\]

Si $X$ es una variedad algebraica, se define el \textbf{grupo de Picard} de $X$, como el grupo abeliano
\[
    \Pic X := \left \{\text{fibrados en recta en $X$}\right \}\big/\cong,
\]
con la estructura de grupo dada por el producto tensorial de fibrados $L \otimes L'$ y con inversa de un elemento $L$ dada por su fibrado dual $L^\vee$.

Cuando $X$ es una variedad algebraica suave e irreducible, se define el \textbf{fibrado en rectas canónico} $\omega_X$ de $X$ como
\[
    \omega_X := \det (\Omega_X^1),
\]
donde $\Omega_X^1$ es el \textit{fibrado cotangente} de $X$. Un \textbf{divisor canónico} $K_X$ es cualquier divisor tal que
\[
    \omega_X \cong \O_X (K_X) \quad \text{en $\Pic X$}.
\]
Similarmente, su fibrado dual $\omega_X^\vee \cong \O_X (-K_X)$ es llamado el \textbf{fibrado en rectas anticanónico}, y $-K_X$ un \textbf{divisor anti-canónico}.


Cuando $X$ sea una variedad algebraica irreducible suave, tenemos un ``diccionario'' entre fibrados en rectas, divisores de Weil, y divisores de Cartier; el cual utilizaremos libremente (cf. \cite[\S 3]{notas_pedro}):
\[
    \begin{tikzcd}
    \Pic X \ar[leftrightarrow]{rrrrrr}{L \mapsto \mathcal L, \text{ con $\mathcal L (U) := H^0 (U, \rest L U)$}} &&&&&& \{\text{haces invertibles}\}/\cong \\
    \\
    \\
    \operatorname{Div} (X) /\operatorname{PDIV} (X) \ar[leftrightarrow]{uuu}{[D] \mapsto \O_X (D)} \ar[leftrightarrow]{rrrrrr}{D = [(f_i, U_i)] \mapsto \hat D = \sum_Y \nu_Y (f_i) Y} &&&&&& \operatorname{WDiv} (X) /\operatorname{PWDiv} (X)
    \end{tikzcd}
\]

\section{Riemann-Roch para curvas}


En \cite[\S 5]{notas_pedro} vimos el Teorema de Riemann-Roch para curvas $X = C$ (Teorema 5.2.8):
\textit{
\begin{quote}
Sea $X = C$ una curva algebraica proyectiva irreducible, de género $g(C) := h^0 (C, \omega_C) = h^1 (C, \O_C)$. Entonces para todo $L \cong \O_C (D) \in \Pic C$, se tiene que
\[
    \chi (C, \O_C (D) ) = \chi (C, \O_C)  + \deg (D).
\]
Equivalentemente,
\[
    h^0 (C, \O_C (D)) - h^0 (C, \O_C (K_C - D)) = \deg (D) + 1 - g(C).
\]
\end{quote}
}

Informalmente, el Teorema de Riemann-Roch dice que para el caso de una variedad proyectiva irreducible $X$ que sea una curva, podemos escribir la característica de Euler-Poincaré de un dividor $D$ en función de invariantes geométricos de $X$ y una cantidad que se define geométricamente en función de $D$ (en el caso $X =C$ su grado). El propósito de estas notas es probar un enunciado similar para $X = S$ una superficie proyectiva suave irreducible.


\section{Número de intersección}

\begin{definition}
Sean $C$ y $C'$ dos curvas irreducibles distintas en una superficie $S$, y $x \in C \cap C'$, notemos por $\O_x$ al anillo local de $S$ en $x$. En $\O_x$, notemos por $f$ una ecuación que define $C$, y similarmente $g$ a otra que define $C'$. Luego definimos la \textbf{multiplicidad de intersección} de $C$ y $C'$ en $x$ como la cantidad
\[
    m_x (C \cap C') := \dim_{k} \O_x / (f,g).
\]
\end{definition}

\begin{obs}
A priori podría suceder que la dimensión de $\O_x/(f,g)$ como $k$-espacio vectorial sea infinita, sin embargo, como $C$ y $C'$ son curvas distintas, resulta que $\dim_k \O_x /(f,g)$ es finita, ya que podemos aplicar el siguiente resultado (cf. \cite[Corolario 11.8]{atiyah2018introductionToCommutativeAlgebra})
\begin{quote}
Sea $k$ un cuerpo algebraicamente cerrado y $B$ un dominio íntegro el cual es una $k$-álgebra finitamente generada. Entonces para todo ideal primo $\mathfrak p$ de $B$,
\[
    \operatorname{height}\: (\mathfrak p) + \dim_{\text{Krull}} (B / \mathfrak p) = \dim_{\text{Krull}} (B).
\]
\end{quote}
\end{obs}

\begin{example}
Cuando $m_x ( C \cap C') = 1$, geométricamente lo que está pasando es que $C$ y $C'$ son \textbf{transversales} en $x$, i.e., $f$ y $g$ forman un sistema local de coordenadas en un entorno abierto de $x$ (por ejemplo, si las tangentes de $C$ y $C'$ en $x$ son distintas). En efecto, notar que $m_x (C \cap C') = 1$ si y solo si $f$ y $g$ generan el ideal maximal $\mathfrak m_x \subset \O_x$. Claramente si $(f,g) = \mathfrak m_x$ se tiene que $\O_x / (f,g) = \O_x / \mathfrak m_x$ es un cuerpo pues estamos cocientando por un ideal maximal. Recíprocamente, si $\O_x /(f,g)$ tiene dimensión $1$, es un cuerpo, y por lo tanto $(f,g)$ es un ideal maximal de $\O_x$; como $(f,g) \subset \mathfrak m_x$ se sigue que de hecho vale la igualdad.
\end{example}

\begin{center}
\centering
\Inkscape{Ejemplo de dos curvas $C$ y $C'$ tales que se cortan transversalmente en $x \in C \cap C'$, i.e., $m_x (C \cap C') = 1$.}{"./figura_interseccion.pdf_tex"}

\end{center}




\begin{definition}
Sean $C$ y $C'$ dos curvas irreducibles distintas en $S$, definimos el \textbf{número de intersección}:
\[
    \langle C, C' \rangle := \sum_{x \in C \cap C'} m_x (C \cap C').
\]
\end{definition}

\begin{obs}
Nuevamente, podría suceder que la cantidad de la derecha sea infinita, pero ya vimos que cada término es finito, y más aún, notemos que la sumatoria tiene una cantidad finita de términos pues, al ser $C$ y $C'$ curvas irreducibles distintas, no se pueden intersecar en más de finitos puntos: $C\cap C'$ es un cerrado de $C$ que no tiene muchas posibilidades, puede tener dimensión $1$ o $0$, y el primer caso no sucede pues $C \neq C'$.
\end{obs}

Recordemos que $C$ y $C'$ se pueden pensar como haces invertibles $\O_S (-C)$ y $\O_S (-C')$ respectivamente, luego definimos el haz
\[
    \O_{C \cap C'} := \O_S \big / \left ( \O_S (-C) \oplus \O_S (-C') \right).
\]
Resulta que este haz es un \textit{haz rascacielos}, concentrado en el conjunto finito $C \cap C'$; en cada uno de estos puntos $x$, tenemos que $(\O_{C \cap C'})_x = \O_x / (f,g)$. Por lo tanto
\[
    (C,C') = h^0 (S, \O_{C\cap C'})  = \chi (S, \O_{C \cap C'}).
\]

\begin{theorem}\label{th:teorema 1}
Para $L$ y $L'$ dos fibrados en rectas de $\Pic S$, definimos
\[
   \langle L , L' \rangle  := \chi (\O_S) - \chi (L^\vee) - \chi (L'^{\vee}) + \chi (L^\vee \otimes L'^{\vee}).
\]
Entonces se tiene que $\langle \cdot, \cdot \rangle : \Pic S \times \Pic S \to \reals$ es una forma simétrice bilineal en el grupo abeliano $\Pic S$, de tal suerte que si $C$ y $C'$ son dos curvas irreducibles distintas en $S$, se tiene que
\[
    \langle \O_S (C), \O_ S (C')\rangle = \langle C,C'\rangle.
\]
\end{theorem}

En otras palabras, el invariante geométrico número de intersección se puede traducir a un invariante cohomológico pensando a las curvas como haces invertibles.

La demostración detallada del teorema se encuentra en \cite{beauville1996complexAlgebraicSurfaces}. Daré solamente un bosquejo:
\begin{proof}[Sketch de la demostración]


\begin{lemma}
\begin{enumerate}[(a)]
\item Sean $s \in H^0 (S, \O_S (C))$ y $s' \in H^0 (S, \O_S(C'))$ secciones no nulas que se anulan en $C$ y $C'$ respectivamente. Entonces la sucesión
\[
    \begin{tikzcd}
    0 \ar{r} & \O_S (-C - C') \ar{r}{(s',-s)} & \O_S (-C) \oplus \O_S (-C') \ar{r}{(s, s')} & \O_S \ar{r} & \O_{C \cap C'} \ar{r} & 0
    \end{tikzcd}
\]
es exacta.
\item Sea $C$ una curva suave irreducible en $S$. Para todo $L \in \Pic S$, se tiene que
\[
    \langle \O_S (C), L \rangle = \deg (\rest L C)
\]
\item Sea $D$ un divisor en $S$, y $H$ una sección por un hiperplano de $S$, entonces existe $n \geq 0$ tal que $D + n H$ es una sección por un hiperplano. En particular, podemos escribir $D \sim A- B$, donde $A$ y $B$ son curvas suaves de $S$ con $A \sim D + n H$ y $B \sim n H$.
\end{enumerate}
\end{lemma}
(El ítem (c) fue visto en clase).

\bigskip

Ahora, el ítem (a) y la aditividad de la característica de Euler-Poincaré nos da $\langle \O_S (C), \O_S (C')\rangle = \langle  C, C'\rangle$, pues desarrollando la definición del lado izquierdo, se puede ver que es igual a $\chi (\O_{C \cap C'}) = h^0 (C \cap C') = \langle C, C'\rangle$. Luego para probar el teorema basta probar la bilinearidad (que la forma es simétrica es obvio). Si $L$ y $L'$ son dos haces invertibles. Por el ítem (c), podemos escribir $L' = \O_S (A - B)$, donde $A$ y $B$ son dos curvas suaves en $S$. Consideremos la expresión
\[
    s(L_1, L_2,L_3) := \langle L_1, L_2 \otimes L_3 \rangle  - \langle L_1, L_2 \rangle  - \langle L_1, L_3 \rangle;
\]
claramente la expresión es simétrica en las tres variables, y el ítem (b) implica que si $L_1 = \O_S (C)$ para una curva irreducible suave $C$ de $S$, entonces $s(L_1,L_2, L_3) = 0$; simétricamente, si $L_2$ o $L_3$ es $\O_S (C)$ también se anula la expresión. Tomando $L_1 = L$, $L_2 = L'$ y $L_3 = \O_S (B)$, despejamos,
\[
    \langle L, L'\rangle  = \langle L, \O_S (A)\rangle - \langle L, \O_S (B)\rangle.
\]
Así, el ítem (b) prueba que $\langle L, L' \rangle $ es lineal en $L$ (ambos términos lo son). La simetría implica que es lineal en la segunda coordenada también.
\end{proof}
En otras palabras, que $\langle \cdot, \cdot\rangle $ sea bilineal en $\Pic S$, implica que:
\begin{enumerate}[(i)]
\item $ \langle L \otimes L', L'' \rangle = \langle L , L'' \rangle  + \langle L', L''\rangle$.
\item $ \langle L^\vee, L'\rangle  = - \langle L, L'\rangle $ y $\langle \O_S, L\rangle  = 0$.
\end{enumerate}
(En la primera coordenada. Similarmente, en la segunda valen estas identidades).

\begin{remark}
Dados dos divisores en $S$, digamos $D$ y $D'$, la gracia del teorema es que podemos calcular el producto $\langle \O_S (D), \O_S (D')\rangle$ reemplazando $D$ y $D'$ por divisores linealmente equivalentes (pues su representante en $\Pic S$ no cambia!).
\end{remark}
Esto tiene dos implicancias:

\begin{proposition}
\begin{enumerate}[(1)]
\item Sea $C$ una curva suave, $f : S \to C$ un morfismo sobreyectivo, $F$ una fibra de $f$. Entonces $\langle \O_S (F), \O_S (F)\rangle  = 0$.
\item Si $S'$ es una superficie y $g : S \to S'$ es un morfismo genéricamente finito de grado $d$, entonces
\[
    \langle \O_S (g^* D), \O_S (g^* D')\rangle  = d \langle  \O_S (D), \O_S (D')\rangle .
\]
\end{enumerate}
\end{proposition}
\begin{proof}
\begin{enumerate}[(1)]
\item Escribamos $F = f^* \{x\}$ para algún $C$. Existe un divisor $A$ en $C$ linealmente equivalente a $x$ tal que $x \not \in A$, con lo cual $F \sim f^* A$. Como $f^* A$ es una combinación lineal de fibras de $f$ distintas de $F$, tenemos que
\[
    \langle \O_S (F), \O_S (F)\rangle  = \langle \O_S (F), f^* A\rangle  = 0.
\]
\item Por el ítem (c), basta probar la fórmula para el caso en el que $D$ y $D'$ son secciones de hiperplanos de $S$. Existe un abierto $U$ de $S'$ tal que las fibras de $g$ tienen grado $d$. Entonces podemos mover $D$ y $D'$ de tal manera que se intersecten transversalmente y la intersección caiga en $U$. Consecuentemente, $g^* D$ y $g^* D'$ se intersectan transversalmente y además $g^* D \cap g^* D' = g^{-1} (D \cap D')$, y se sigue el resultado.
\end{enumerate}
\end{proof}

\begin{example}[Teorema de Bezout]
Tomemos $S = \mathbb{P}^2$. Recordemos que $\Pic {\mathbb{P}^2} \cong \integers$: más precisamente, toda curva de grado $d$ es linealmente equivalente a $d L$ para una recta $L$. Así, sean $C$ y $C'$ dos curvas de grado $d$ y $d'$, y sean $L$ y $L'$ dos rectas distintas; como $C \sim d L$ y $C' \sim d' L'$, el Teorema \ref{th:teorema 1} implica el Teorema de Bezout:
\[
    \langle C, C' \rangle  = \langle \O_S (C), \O_S (C')\rangle  = \langle d L, d L' \rangle = d d' \langle L, L'\rangle  = d d'.
\]
\end{example}



\section{Riemann-Roch para superficies}

Recordemos Teorema de dualidad de Serre (cuya demostración vimos en \cite[Teorema 5.1.3]{notas_pedro}):

\begin{theorem}[Dualidad de Serre]\label{th:dualidad de serre}
Sea $X$ una variedad algebraica proyectiva suave e irreducible de dimensión $n$, y sea $\omega_X = \det (\Omega_X^1)$. Entonces, para todo fibrado vectorial $E \to X$,
\[
    H^i (X, E) \cong H^{n-i} (X, E^\vee \otimes \omega_X)^{\vee}.
\]
En particular, $h^i (X, E) = h^i (X, E^\vee \otimes \omega_X)$.
\end{theorem}
En nuestro caso $X = S$, deducimos que $\chi (L) = \chi (\omega_S \otimes L^\vee)$ por la definición de número de Euler-Poincaré.

Gracias a esto podemos probar el Teorema de Riemann-Roch para superficies:

\begin{theorem}[Riemann-Roch para superficies]\label{th:riemann-roch para superficies}
Para todo $L \in \Pic S$, se tiene
\[
    \chi (L) = \chi (\O_S) + \frac 1 2 \big ( \langle L, L\rangle  - \langle L, \omega_S\rangle  \big ).
\]
\end{theorem}
\begin{proof}
Primero calculemos por definición:
\[
    \langle L^\vee, L \otimes \omega_S^\vee \rangle  := \chi (\O_S) - \chi (L) - \chi (\omega_S \otimes L^\vee) + \chi (\omega_S).
\]
Por dualidad de Serre \ref{th:dualidad de serre}, $\chi (\omega_S) = \chi (\O_S)$ y $\chi (\omega_S \otimes L ^\vee) = \chi (L)$, y por lo tanto
\[
    \langle L^\vee, L \otimes \omega_S^\vee \rangle = 2 \left ( \chi (\O_S) - \chi (L) \right).
\]
Por otro lado, aplicando bilinealidad (\ref{th:teorema 1}) del lado izquierdo:
\begin{align*}
\langle L^\vee, L \otimes \omega_S^\vee \rangle  &= \langle L^\vee, L\rangle + \langle L^\vee, \omega_S^\vee \rangle \\
&= - \langle L , L\rangle  + \langle L, \omega_S\rangle.
\end{align*}
Juntando ambas igualdades obtenemos el teorema.
\end{proof}


\begin{remark}
Podemos reescribir el enunacio de Riemann-Roch en término de divisores. Sea $h^i (D) := h^i(S, \O_S (D))$ y $K_S$ un divisor canónico, i.e., $\O_S (K_S) = \omega_S$. En este lenguaje la dualidad de Serre \ref{th:dualidad de serre} nos queda $h^i (D) = h^{n-i} (K_S - D)$ con $n = \dim (S) = 2$. Así, Riemann-Roch \ref{th:riemann-roch para superficies} se reescribe como:
\[
    h^0 (D) + h^0 (K-D) - h^1 (D) = \chi (\O_S) + \frac 1 2 \big ( \langle \O_S (D), \O_S (D) \rangle - \langle \O_S (D), \O_S (K_S) \rangle \big ).
\]

Usualmente no tendremos información de $h^1 (D)$, aún así Riemann-roch nos provee de la siguiente desigualdad útil:
\[
    h^0 (D) + h^0 (K_S - D) \geq \chi (\O_S) + \frac 1 2 \big ( \langle \O_S (D), \O_S (D) \rangle  - \langle  \O_S (D), \O_S (K)\rangle  \big ).
\]
\end{remark}

\bigskip

Sea $g(C) := h^1 (C, \O_C)$ el género de una curva $C$ suave irreducible en $S$, entonces podemos deducir la siguiente fórmula a partir de Riemann-Roch:
\begin{theorem}[Fórmula del Género]\label{th:formula del genero}
Sea $C$ una curva irreducible suave en $S$. Entonces
\[
    g(C) = 1 + \frac 1 2 \big ( \langle \O_S (C), \O_S (C) \rangle  - \langle \O_S (C), \O_S (K_S) \rangle  \big ).
\]
\end{theorem}
\begin{proof}
Primero, tenemos la sucesión exacta
\[
    \begin{tikzcd}
    0 \ar{r} & \O_S (-C) \ar{r} & \O_S \ar{r} & \O_C \ar{r} & 0,
    \end{tikzcd}
\]
la cual utilizando la aditividad de la característica de Euler-Pincaré nos da
\[
    \chi (\O_C) = \chi (\O_S) - \chi (\O_S (-C)).
\]
Como $ \chi (\O_C ) := 1 - g(C)$, nos queda,
\[
    1 - g(C) = \chi (\O_S) - \chi (\O_S (-C)).
\]
Finalmente, obtenemos la fórmula aplicando Riemann-Roch con $L = \O_S (-C)$, pues nos dice que
\begin{align*}
    \chi (\O_S (-C)) &= \chi (\O_S) + \frac 1 2 \big ( \langle \O_S (-C), \O_S (-C) \rangle - \langle \O_S (-C), \O_S (K_S) \rangle \big) \\
    &= \chi (\O_S) + \frac 1 2 \big ( \langle \O _S (C), \O_S (C)\rangle + \langle \O_S (C), \O_S (K_S) \rangle  \big ),
\end{align*}
con lo cual podemos reemplazar esto arriba y despejar $g(C)$.
\end{proof}


\begin{example}
Cuando $S = \mathbb{P}^2$, si $C$ es una curva irreducible suave de grado $d$ en $S$, como $\Pic S \cong \integers$, podemos elegir cualquier fibrado en rectas $L$ y se tiene que $C \sim d L$. Recordar que $\omega_S = -3 L'$ para cualquier otra recta $L'$ distinta de $L$. Entonces la Fórmula del Género \ref{th:formula del genero} implica que
\begin{align*}
g(C) = 1 + \frac 1 2 \big ( d^2 \langle L, L \rangle - 3 d \langle L, L' \rangle \big ).
\end{align*}
Ahora, $\langle L, L' \rangle = 1$ porque $L$ y $L'$ son dos rectas distintas en $\mathbb{P}^2$; por otro lado, $\langle L, L \rangle := \chi (\O_{\mathbb{P}^2}) - 2 \chi (L^\vee) + \chi (L^\vee \otimes L^\vee)$, pero cada característica de Euler-Poincaré se puede calcular conociendo los coeficientes de cohomología (ver \cite[Teorema 4.2.1]{notas_pedro}): $\chi (\O_{\mathbb{P}^2}) = 1$, $\chi (L^\vee) = 0$, $\chi (L^\vee \otimes L^\vee) = 0$, ya que podemos notar que $L^\vee \cong \O_{\mathbb{P}^2} (-1)$ y $L^\vee \otimes L^\vee \cong \O_{\mathbb{P}^2} (-2)$. Consecuentemente,
\[
    g(C) = 1 + \frac 1 2 \big ( d^2 - 3d \big) = \frac{(d-1)(d-2)}{2}.
\]

En particular, si $C$ es una curva elíptica, entonces $g(C) = 1$ porque $d = 3$.
\end{example}





\bibliographystyle{amsalpha}
\bibliography{main}



\end{document}
